% -*- root: Document.tex -*-

\subsection{Manage a cluster}
\label{subsec:clustersetup:manage}

We are assuming that your hosts are all reachable by SSH from the CLI.
You can ensure that by running:

\begin{lstlisting}[language=bash, basicstyle=\small]
$ ./remote.rb ping
\end{lstlisting}

Once the configuration of the hosts to use for creating the cluster has been given, here are the few steps needed to create a cluster:

\begin{itemize}
  \item \emph{Create nodes}: this command will configure \textsc{Docker} on the nodes to use in the \textsc{Docker Swarm} cluster, by resetting the configuration file of the \textsc{Docker} service of each host, like explained in Subsection~\ref{subsec:docker_config}:
    \begin{lstlisting}[language=bash, basicstyle=\small]
  $ ./remote.rb config-docker
    \end{lstlisting}
  \item \emph{Create the cluster}: this command will deploy the \textsc{Docker Swarm} cluster:
    \begin{lstlisting}[language=bash, basicstyle=\small]
  $ ./remote.rb create
    \end{lstlisting}
  \item \emph{Create an overlay network}: this command will create a \textsc{Dockr} overlay network:
    \begin{lstlisting}[language=bash, basicstyle=\small]
  $ ./remote.rb network
    \end{lstlisting}
\end{itemize}

All these commands can be run at once:

\begin{lstlisting}[language=bash, basicstyle=\small]
    $ ./remote.rb ping config-docker create network
\end{lstlisting}

You can ensure that your cluster have been well created by running:

\begin{lstlisting}[basicstyle=\small]
$ ./remote.rb test
\end{lstlisting}

If you want to recreate a cluster, use the command \texttt{remove} before, \textit{e.g.}:

\begin{lstlisting}[language=bash, basicstyle=\small]
    $ ./remote.rb remove
\end{lstlisting}

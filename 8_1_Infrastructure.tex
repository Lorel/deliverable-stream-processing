% -*- root: Document.tex -*-

\section{Infrastructure}
\label{sec:infrastructure}

The \textsc{Docker Swarm} cluster can be built with some hosts using:

\begin{itemize}
  \item \textsc{Ubuntu 16.04.2 LTS};
  \item \textsc{Docker} v. 1.13.0 (cf. Subsection~\ref{subsec:docker_install});
  \item \textsc{Docker Swarm} v. 1.2.5 with the \textsc{Docker} image labeled \texttt{swarm:1.2.5}\footnote{\textsc{Docker Swarm} on \textsc{DockerHub}: \url{https://hub.docker.com/_/swarm/}.};
  \item the key-value store \textsc{Consul} using its latest \textsc{Docker} image\footnote{\textsc{Consul} on \textsc{DockerHub}: \url{https://hub.docker.com/r/progrium/consul/}.}.
\end{itemize}

\newpage

\subsection{Install \textsc{Docker} v. 1.13.0}
\label{subsec:docker_install}

Run the following commands to install \textsc{Docker} v. 1.13.0 on \textsc{Ubuntu 16.04.2 LTS}:

\begin{itemize}
  \item
    Add the APT key:
    \begin{lstlisting}[language=bash, basicstyle=\small]
$ sudo apt-key adv --keyserver hkp://p80.pool.sks-keyservers.net:80 \
  --recv-keys 58118E89F3A912897C070ADBF76221572C52609D
    \end{lstlisting}
  \item
    Add the \textsc{Docker} repository in the sources list of APT:
    \begin{lstlisting}[language=bash, basicstyle=\small]
$ sudo bash \
  -c 'echo "deb https://apt.dockerproject.org/repo ubuntu-xenial main" \
  > /etc/apt/sources.list.d/docker.list'
    \end{lstlisting}
  \item
    Update APT:
    \begin{lstlisting}[language=bash, basicstyle=\small]
$ sudo apt-get update
    \end{lstlisting}
  \item
    Be sure to have this package installed:
    \begin{lstlisting}[language=bash, basicstyle=\small]
$ sudo apt-get install linux-image-extra-$(uname -r)
    \end{lstlisting}
  \item
    You can then list all available package of \textsc{Docker} and should find out the version 1.13.0:
    \begin{lstlisting}[language=bash, basicstyle=\small]
$ apt-cache policy docker-engine
    \end{lstlisting}
  \item
    Install \textsc{Docker} v. 1.13.0:
    \begin{lstlisting}[language=bash, basicstyle=\small]
$ sudo apt-get install docker-engine=1.13.0-0~ubuntu-xenial
    \end{lstlisting}
\end{itemize}

\newpage

\subsection{Configure \textsc{Docker}}
\label{subsec:docker_config}

\textsc{Docker} has to be configured correctly on each host with the key-value store \textsc{Consul}.
If your host is using the boot manager \textsc{systemd} (like \textsc{Ubuntu 16.04.2 LTS} does by default), you can edit the file \texttt{/lib/systemd/system/docker.service} and change the value of \texttt{ExecStart} to \texttt{/usr/bin/docker daemon} the following options:

\begin{itemize}
  \item \texttt{-H tcp://0.0.0.0:2375}: bind \textsc{Docker} on \texttt{0.0.0.0} for it to be reachable from outside the host;
  \item \texttt{--cluster-store=consul://172.16.0.2:8500}: provide the IP address (\textit{e.g.} here \texttt{172.16.0.2}) and the port (\textit{e.g.} here \texttt{8500}) of the key-value store \textsc{Consul};
  \item \texttt{--cluster-advertise=enp0s31f6:2375}: provide the network interface (\textit{e.g.} here \texttt{enp0s31f6}) and the port (\textit{e.g.} here \texttt{2375}) of the \textsc{Docker} daemon;
  \item \texttt{--label type="sgx"}: the label to give to the \textsc{Docker} daemon in the \textsc{Docker Swarm} cluster (it is mandatory to target the correct host when launching the containers with \textsc{Docker Compose}, as we will explain later).
\end{itemize}

Note that all this configuration can be defined easily using the tiny CLI tool \texttt{remote} (cf. Section~\ref{sec:clustersetup}).

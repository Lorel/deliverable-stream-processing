% -*- root: Document.tex -*-

\section{\SS{} Usage}
\label{sec:securestreamusage}

The code of \SS{} can be retrieved from the repository \securestreamrepo{}.

The main idea behind \SS{} is to run each steps of a processing pipeline (\textit{e.g.} Listing~\ref{rx-processing-pipeline-example}) on different nodes of a cluster.
To do this, \SS{} provides an API built on top of \textsc{RxLua}, the Reactive Extensions for Lua\footnote{\textsc{RxLua}, Reactive Extensions for Lua: \url{https://github.com/bjornbytes/RxLua}.} (see Subsection~\ref{subsec:rxluaapi}), and relies on \zmq{} by the implementation of an extension of the \textsc{RxLua} (see Subsection~\ref{subsec:rxluazmq}).
Finally, \SS{} provides also another extension of \textsc{RxLua} to use seamlessly the \luavm{} modified to process some code in an SGX enclave (see Subsection~\ref{subsec:rxluasgx}).


\subsection{\textsc{RxLua} API}
\label{subsec:rxluaapi}

\textsc{RxLua} is an API to implement a processsing pipeline in the way of the dataflow programming paradigm.
This library permits to describe a processsing pipeline like shown in Listing~\ref{rx-processing-pipeline-example}.

\begin{lstlisting}[language=LUA,caption={Example of process pipeline with RxLua.},label=rx-processing-pipeline-example]
local Rx = require 'rx'
local csv = require 'csv'

local input = 'data.csv'
local output = 'output.log'
local start = os.time()
local results = {}

io.output(file)

Rx.Subject.fromFileByLine(input)
  :map(
    function(value)
      if not value then return {} end

      local array = csv.parse(value)
      local event = {}
      event.uniquecarrier = array[9]
      event.arrdelay = array[15]

      return event
    end
  )
  :filter(
    function(event)
      return (tonumber(event.arrdelay) or 0) > 0
    end
  )
  :reduce(
    function(accumulator, event)
      carrier = accumulator[event.uniquecarrier] or {}
      accumulator[event.uniquecarrier] = { count = (carrier.count or 0) + 1, total = (carrier.total or 0) + event.arrdelay }

      return accumulator
    end, {}
  )
  :subscribe(
    function(datas)
      for k,v in pairs(datas) do
        carrier = results[k] or {}
        results[k] = { count = (carrier.count or 0) + v.count, total = (carrier.total or 0) + v.total }
      end
    end,
    function(error)
      print(error)
    end,
    function()
      io.write((os.time() - start) .. "\n")
      io.close(ouput)

      for k,v in pairs(results) do
        print('Delays for carrier ' .. k .. ' -> ' .. (v.count == 0 and 0 or (v.total / v.count)) .. ' average mins - ' .. math.tointeger(v.count) .. " delayed flights")
      end

      ZmqRx.sendZmqCompleted()
      print('completed!')
    end
  )
\end{lstlisting}


\newpage

\subsection{\textsc{RxLua} extension for \zmq{}}
\label{subsec:rxluazmq}

We have implemented a layer on top of \textsc{RxLua} to define a communication channel using \zmq{} between two steps of a processing pipeline.
This layer has been written in the file \texttt{zmq-rx.lua} and provides 2 functions:

\begin{itemize}
  \item \textbf{Rx.Observable.fromZmqSocket(\texttt{socket})}: defines a \zmq{} where to get a data stream;
  \item \textbf{Rx.Observable:subscribeToSocket(\texttt{socket})}: defines a \zmq{} where to send a data stream.
\end{itemize}

The parameter \texttt{socket} is a formatted string giving the protocol, the address and the port of the socket (\textit{e.g.} \texttt{'tcp://localhost:5555'}).

Using this layer for \zmq{}, we can split the 5 steps of the processsing pipeline shown in Listing~\ref{rx-processing-pipeline-example} on 5 independant files of \lua{} code:

\begin{itemize}[leftmargin=*]
  \item The data input steam:
    \begin{lstlisting}[language=LUA,caption={Process pipeline with Zmq-RxLua: data stream.},label=rx-processing-data-stream]
local ZmqRx = require 'zmq-rx'

local input = 'data.csv'
local to_socket = 'tcp://localhost:5555'

ZmqRx.Subject.fromFileByLine(input)
  :subscribeToSocket(to_socket) -- send to 'tcp://localhost:5555'
\end{lstlisting}

  \item The mapper from \textsc{CSV} line to data event:
    \begin{lstlisting}[language=LUA,caption={Process pipeline with Zmq-RxLua: mapper.},label=rx-processing-map-csv-to-event]
local ZmqRx = require 'zmq-rx'
local csv = require 'csv'

local from_socket = 'tcp://localhost:5555'
local to_socket = 'tcp://localhost:5556'


ZmqRx.Subject.fromZmqSocket(from_socket) -- receive from 'tcp://localhost:5555'
  :map(
    function(value)
      if not value then return {} end

      local array = csv.parse(value)
      local event = {}
      event.uniquecarrier = array[9]
      event.arrdelay = array[15]

      return event
    end
  )
  :subscribeToSocket(to_socket) -- send to 'tcp://localhost:5556'
\end{lstlisting}

  \item The event filter:
    \begin{lstlisting}[language=LUA,caption={Process pipeline with Zmq-RxLua: filter.},label=rx-processing-filter-event]
local ZmqRx = require 'zmq-rx'

local from_socket = 'tcp://localhost:5556'
local to_socket = 'tcp://localhost:5557'

ZmqRx.Subject.fromZmqSocket(from_socket) -- receive from 'tcp://localhost:5556'
  :filter(
    function(event)
      return (tonumber(event.arrdelay) or 0) > 0
    end
  )
  :subscribeToSocket(to_socket) -- send to 'tcp://localhost:5557'
\end{lstlisting}

  \item The events reducer:
    \begin{lstlisting}[language=LUA,caption={Process pipeline with Zmq-RxLua: reducer.},label=rx-processing-reduce-events]
local ZmqRx = require 'zmq-rx'

local from_socket = 'tcp://localhost:5557'
local to_socket = 'tcp://localhost:5558'

ZmqRx.Subject.fromZmqSocket(from_socket) -- receive from 'tcp://localhost:5557'
  :reduce(
    function(accumulator, event)
      carrier = accumulator[event.uniquecarrier] or {}
      accumulator[event.uniquecarrier] = { count = (carrier.count or 0) + 1, total = (carrier.total or 0) + event.arrdelay }

      return accumulator
    end, {}
  )
  :subscribeToSocket(to_socket) -- send to 'tcp://localhost:5558'
\end{lstlisting}

  \item The results printer:
    \begin{lstlisting}[language=LUA,caption={Process pipeline with Zmq-RxLua: printer.},label=rx-processing-print-result]
local ZmqRx = require 'zmq-rx'

local from_socket = 'tcp://localhost:5558'

local output = 'output.log'
local start = os.time()
local results = {}

io.output(file)

ZmqRx.Subject.fromZmqSocket(from_socket) -- receive from 'tcp://localhost:5558'
  :subscribe(
    function(datas)
      for k,v in pairs(datas) do
        carrier = results[k] or {}
        results[k] = { count = (carrier.count or 0) + v.count, total = (carrier.total or 0) + v.total }
      end
    end,
    function(error)
      print(error)
    end,
    function()
      io.write((os.time() - start) .. "\n")
      io.close(output)

      for k,v in pairs(results) do
        print('Delays for carrier ' .. k .. ' -> ' .. (v.count == 0 and 0 or (v.total / v.count)) .. ' average mins - ' .. math.tointeger(v.count) .. " delayed flights")
      end

      ZmqRx.sendZmqCompleted()
      print('completed!')
    end
  )
\end{lstlisting}

\end{itemize}

Between each step of this processing pipeline, data are streamed through some routers using the code from the file \texttt{router.lua}.
These routers forward the data from a step to another one, without any process on them, using some \zmq{} channels.

\newpage

\subsection{\textsc{RxLua} extension for SGX}
\label{subsec:rxluasgx}

\SS{} can execute the code of a function \texttt{map}, \texttt{filter}, or \texttt{reduce} inside an SGX enclave.
To do that, it uses the primitives provided by the modified \luavm{} (which are \texttt{sgxprocess}, \texttt{sgxencrypt}, and \texttt{sgxdecrypt}, like described in Chapter~\ref{chap:proto}).
These primitives are called from the library implemented in the file \texttt{sgx.lua}.
This library provides also some mocks for these primitives, in case the code has to be executed in a regular \luavm{}.

\SS{} provides finally a layer built on top of \textsc{RxLua}, and which provides the function \texttt{mapSGX}, \texttt{filterSGX}, and \texttt{reduceSGX} with whom the steps of mapping, filtering and reducing (described in Subsection~\ref{subsec:rxluazmq}) can be rewritten as following to be executed inside a SGX enclave:

\begin{itemize}[leftmargin=*]
  \item The mapper from \textsc{CSV} line to data event:
    \begin{lstlisting}[language=LUA,caption={Process pipeline with Zmq-RxLua using SGX: mapper.},label=rx-processing-sgx-map-csv-to-event]
local ZmqRx = require 'sgx-rx'
local csv = require 'csv'

local from_socket = 'tcp://localhost:5555'
local to_socket = 'tcp://localhost:5556'


ZmqRx.Subject.fromZmqSocket(from_socket) -- receive from 'tcp://localhost:5555'
  :mapSGX(
    function(value)
      if not value then return {} end

      local array = csv.parse(value)
      local event = {}
      event.uniquecarrier = array[9]
      event.arrdelay = array[15]

      return event
    end
  )
  :subscribeToSocket(to_socket) -- send to 'tcp://localhost:5556'
\end{lstlisting}

  \item The event filter:
    \begin{lstlisting}[language=LUA,caption={Process pipeline with Zmq-RxLua using SGX: filter.},label=rx-processing-sgx-filter-event]
local ZmqRx = require 'sgx-rx'

local from_socket = 'tcp://localhost:5556'
local to_socket = 'tcp://localhost:5557'

ZmqRx.Subject.fromZmqSocket(from_socket) -- receive from 'tcp://localhost:5556'
  :filterSGX(
    function(event)
      return (tonumber(event.arrdelay) or 0) > 0
    end
  )
  :subscribeToSocket(to_socket) -- send to 'tcp://localhost:5557'
\end{lstlisting}

  \item The events reducer:
    \begin{lstlisting}[language=LUA,caption={Process pipeline with Zmq-RxLua using SGX: reducer.},label=rx-processing-sgx-reduce-events]
local ZmqRx = require 'sgx-rx'

local from_socket = 'tcp://localhost:5557'
local to_socket = 'tcp://localhost:5558'

ZmqRx.Subject.fromZmqSocket(from_socket) -- receive from 'tcp://localhost:5557'
  :reduceSGX(
    function(accumulator, event)
      carrier = accumulator[event.uniquecarrier] or {}
      accumulator[event.uniquecarrier] = { count = (carrier.count or 0) + 1, total = (carrier.total or 0) + event.arrdelay }

      return accumulator
    end, {}
  )
  :subscribeToSocket(to_socket) -- send to 'tcp://localhost:5558'
\end{lstlisting}

\end{itemize}

Note that the code of the function passed to each step of the processing pipeline does not change between the regular version and the one usig SGX.
Only the name of the required file for \texttt{ZmqRx} and the name of the used functions are different.

\newpage

\subsection{\textsc{Docker} images for \SS{}}
\label{subsec:ssdockerimages}

\SS{} executes each step of the processing pipeline and each router in some \textsc{Docker} containers using either the image \dockerimagelua{} or the one \dockerimageluasgx{}, depending if the code to execute uses the regular \luavm{} or the one modified for SGX.
These images have to be provided to the configuration file used by \textsc{Docker Compose} to run all the containers of the pipeline on the \textsc{Docker Swarm} cluster (see Listing~\ref{docker-compose-config} for a complete configuration).

For each container, the following values are mandatory:

\begin{itemize}
  \item \textbf{image}: the \textsc{Docker} image to use, either \dockerimagelua{} or \dockerimageluasgx{};
  \item \textbf{entrypoint}: the command to run when the container is started, either \texttt{lua} for using the regular \luavm{}, or \texttt{./start.sh} for using the one modified for SGX, each of them followed by the name of the file with the code to execute for this node (\textit{i.e.} the one for the corresponding step of the processing pipeline);
  \item \textbf{environment}: it expects a list of the environment variables to pass to the container, typically \texttt{TO} and \texttt{FROM} with the address for the corresponding sockets, and a constraint used by the \textsc{Docker Swarm} scheduler to launch the given container on a specific host of the cluster;
  \item \textbf{env\_file}: pass a file to the container with some environment wariables, typically the ones for setting the logger and the endpoint for the common controller of the pipeline (here the printer) - convenient for giving some variables shared by all containers;
  \item \textbf{volumes}: the list of the volumes to mount into the container - here have to be mount the directory with the common code of \SS{} and the file with the code of the processing pipeline step to be executed by the current container;
  \item \textbf{networks}: the list of the networks joined by the container - it must contain the overlay network shared by all containers, and defined at the end of this \textsc{Docker Compose} file.
\end{itemize}


\begin{lstlisting}[language=YAML,caption={Configuration file example for \textsc{Docker Compose}.},label=docker-compose-config][t]
version: '2'

services:
  printer:
    image: "lorel/zmqrxlua:lua"
    entrypoint: lua print-results.lua
    environment:
      - FROM=tcp://routerreduceprinter:5562
      - OUTPUT=$RESULT_OUTPUT
      - "constraint:type==printer"
    env_file: global.env
    volumes:
      - "/path/to/securestream/code:/root/worker"

    networks:
      xp:
        aliases:
          - controller

  routerreduceprinter:
    image: "lorel/zmqrxlua:lua"
    hostname: routerreduceprinter
    entrypoint: lua router.lua
    environment:
      - TO=tcp://*:5562
      - FROM=tcp://*:5561
      - "constraint:type==worker"
    env_file: global.env
    volumes:
      - "/path/to/securestream/code:/root/worker"
    networks:
      - xp

  reduce:
    image: "lorel/zmqrxlua:lua"
    entrypoint: lua reduce-events.lua
    environment:
      - TO=tcp://routerreduceprinter:5561
      - FROM=tcp://routerfilterreduce:5560
      - "constraint:type==worker"
    env_file: global.env
    volumes:
      - "/path/to/securestream/code:/root/worker"
    networks:
      - xp

  routerfilterreduce:
    image: "lorel/zmqrxlua:lua"
    hostname: routerfilterreduce
    entrypoint: lua router.lua
    environment:
      - TO=tcp://*:5560
      - FROM=tcp://*:5559
      - "constraint:type==worker"
    env_file: global.env
    volumes:
      - "/path/to/securestream/code:/root/worker"
    networks:
      - xp

  filter:
    image: "lorel/zmqrxlua:lua"
    entrypoint: lua filter-event.lua
    environment:
      - TO=tcp://routerfilterreduce:5559
      - FROM=tcp://routermapperfilter:5558
      - "constraint:type==worker"
    env_file: global.env
    volumes:
      - "/path/to/securestream/code:/root/worker"
    networks:
      - xp

  routermapperfilter:
    image: "lorel/zmqrxlua:lua"
    hostname: routermapperfilter
    entrypoint: lua router.lua
    environment:
      - TO=tcp://*:5558
      - FROM=tcp://*:5557
      - "constraint:type==worker"
    env_file: global.env
    volumes:
      - "/path/to/securestream/code:/root/worker"
    networks:
      - xp

  mapper:
    image: "lorel/zmqrxlua:lua"
    entrypoint: lua map-csv-to-event.lua
    environment:
      - TO=tcp://routermapperfilter:5557
      - FROM=tcp://routerdatamapper:5556
      - "constraint:type==worker"
    env_file: global.env
    volumes:
      - "/path/to/securestream/code:/root/worker"
    networks:
      - xp

  routerdatamapper:
    image: "lorel/zmqrxlua:lua"
    hostname: routerdatamapper
    entrypoint: lua router.lua
    environment:
      - TO=tcp://*:5556
      - FROM=tcp://*:5555
      - "constraint:type==worker"
    env_file: global.env
    volumes:
      - "/path/to/securestream/code:/root/worker"
    networks:
      - xp

  data:
    image: "lorel/zmqrxlua:lua"
    entrypoint: lua data-stream.lua
    environment:
      - TO=tcp://routerdatamapper:5555
      - "constraint:type==data"
      - DATA_FILE=data/2008.csv
    env_file: global.env
    volumes:
      - "/path/to/securestream/code:/root/worker"
    networks:
      - xp


networks:
  xp:
    external:
      name: default_network
\end{lstlisting}


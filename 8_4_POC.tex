\section{How to run the Proof-of-Concept}
\label{sec:pocusage}

You can find here all the steps to run the proof-of-concept of \SS{}.


\subsection{Create a \textsc{Docker Swarm} cluster and its overlay network}
\label{subsec:pocusage:init-cluster}

After a correct configuration of the remote CLI (see Subsection~\ref{subsec:clustersetup:config} for more explanations), use this one to configure a \textsc{Docker Swarm} cluster and its overlay network:

\begin{lstlisting}[language=bash, basicstyle=\small]
$ ./remote.rb ping config-docker create network
\end{lstlisting}


\subsection{Import \SS{} code on each node}
\label{subsec:pocusage:import-code}

The proof-of-concept is based of data which have to be downloaded on the nodes of the cluster where the data streams will be run.
You have to indicate in the file \texttt{config.yml} of the remote CLI which nodes need to download this code by using the attributes \texttt{roles} for that:

\begin{lstlisting}[language=YAML,caption={Configure some roles for a node of the cluster},label=node-roles-config][t]
cluster:
  ...

  nodes:
    -
      ip: 172.16.0.4
      name: 'sgx-1'
      network_if: 'enp0s31f6'
      type: sgx
      roles:
        - data1
        - data2
        - data3
        - data4
\end{lstlisting}

You have also to provide in the same file some details about the repository of \SS{}:

\begin{lstlisting}[language=YAML,caption={Configure the repository of \SS{}},label=repo-config][t]
poc:
  working_dir: '/home/ubuntu/zmqrxlua'
  project_name: 'zmqrxlua-poc'
  experiment_path: 'experiment'
  remote_repo: repo_url
\end{lstlisting}

Give as \texttt{remote\_repo} value the address of the repository of \SS{}: \securestreamrepo{}.

After a correct configuration of the remote CLI, use this one to import the code of \SS{}  on each node:

\begin{lstlisting}[language=bash, basicstyle=\small]
$ ./remote.rb init-xp
\end{lstlisting}


\subsection{Run XP on the cluster}
\label{subsec:pocusage:run-xp}

Finally, you only have to connect to the host of your \textsc{Docker Swarm} manager, go to the directory \texttt{experiment/test} of the cloned repository, and execute the script \texttt{run.sh}.

% -*- root: Document.tex -*-

\section{Cluster Setup}
\label{sec:clustersetup}

A tiny CLI tool has been implemented to setup easily a \textsc{Docker Swarm} cluster and its overlay network.
This one can be found in the directory \texttt{remote} of the repository \securestreamrepo{}.
Once the configuration for the different hosts used to build the cluster is given (cf. Subsection~\ref{subsec:clustersetup:config}), the CLI can be used to build a \textsc{Docker Swarm} cluster (cf. Subsection~\ref{subsec:clustersetup:manage}).
Note that the example of configuration given here is present in the file \texttt{config.yml.example}: you can copy this file to \texttt{config.yml} and modify it with your own configuration.

When the CLI is called without any command, it returns the list of the available ones:

\begin{lstlisting}[basicstyle=\small]
$ ./remote.rb
Welcome to the remote manager !

Version: 0.1.0
Use one or several (you can chain them) of the following commands:

- help		 Print this usage notice
- version	 Print used version of Swarm
- ping		 Check if connexions for each VM are well configured
- create	 Create cluster
- network	 Create Docker overlay network
- hostnames	 Set node hostnames
- config-docker	 Upgrade Docker with the last release, and configure
- test		 Run hello world on each Docker node using the Swarm manager
- init-xp	 Initialize nodes for experimental POC by getting repository

For example:	 ./remote.rb ping config_docker create test network
\end{lstlisting}

\newpage

% -*- root: Document.tex -*-

\subsection{Cluster Configuration}
\label{subsec:clustersetup:config}

The configuration has to be given into the YAML file \texttt{config.yml} at the root of tool's code.
This file has to be build like in the example~\ref{remote-swarm-manager-config}.
All configurations options are detailed below.

% \begin{minipage}{\linewidth} %avoid splitting
% \hspace*{-\parindent}
\begin{lstlisting}[language=YAML,caption={Configuration file example for configuring the \textsc{Docker Swarm} cluster using the CLI.},label=remote-swarm-manager-config][t]
ssh:
  user: 'host_user'
  identity_file: 'path/to/ssh_keys/id_rsa'
  public_key_file: 'path/to/ssh_keys/id_rsa.pub'
  proxy:
    user: 'proxy_user'
    host: 'proxy_host'

cluster:
  manager: 172.16.0.2
  manager_docker_port: 2380
  node_docker_port: 2380
  consul_ip: 172.16.0.3
  consul_port: 8500
  network_name: default_network

  nodes:
    -
      ip: 172.16.0.4
      name: 'sgx-1'
      network_if: 'enp0s31f6'
      type: sgx
      roles:
        - sgx-worker
    -
      ip: 172.16.0.5
      name: 'regular-1'
      network_if: 'enp0s31f6'
      type: regular
      roles:
        - regular-worker

swarm:
  image: swarm:1.2.0
  strategy: spread
\end{lstlisting}
% \end{minipage}

\subsubsection{SSH options}

The CLI is using the protocol \textsc{SSH} to access to the different hosts of your cluster.

\begin{itemize}
  \item \texttt{user} (\emph{mandatory}): the user you want to use to connect to hosts;
  \item \texttt{identity\_file} (\emph{mandatory}): the path to the private key to use to connect to hosts;
  \item \texttt{public\_key\_file} (\emph{optional}): the path to the public key to use to connect to hosts - required only for using CLI's feature for exporting a public key on remote hosts;
  \item \texttt{proxy} (\emph{optional}): the proxy to use to reach the different hosts - required only to use a proxy, it will use the same key than the one for hosts.
\end{itemize}

\subsubsection{Cluster options}

Here are defined all the parameters related to the infrastructure of the cluster:

\begin{itemize}
  \item \texttt{manager} (\emph{mandatory}): IP address of host where the \textsc{Docker Swarm} manager will be deployed;
  \item \texttt{manager\_docker\_port} (\emph{optional}): port where the \textsc{Docker Swarm} manager will listen (default value: \texttt{2381});
  \item \texttt{node\_docker\_port} (\emph{optional}): port where the \textsc{Docker Swarm} nodes will listen (default value: \texttt{2375});
  \item \texttt{consul\_ip} (\emph{mandatory}): IP address of host where the \textsc{Consul} service will be deployed;
  \item \texttt{consul\_port} (\emph{optional}): port where the \textsc{Consul} service will listen (default value: \texttt{8500});
  \item \texttt{network\_name} (\emph{optional}): name of the \textsc{Docker} overlay network (default value: \texttt{default\_network});
  \item \texttt{nodes} (\emph{mandatory}): the list of the \textsc{Docker Swarm} nodes to set in the cluster, with for each:
  \begin{itemize}
    \item \texttt{ip} (\emph{mandatory}): IP address of the host;
    \item \texttt{cpu} (\emph{mandatory}): number of CPUs of the host;
    \item \texttt{network\_if} (\emph{mandatory}): name of the network interface the host use over the cluster;
    \item \texttt{type} (\emph{mandatory}): the value to give to the label \texttt{type} which will be set on the \textsc{Docker} daemon of the current node;
    \item \texttt{roles} (\emph{optional}): a list of roles given as strings for the current node (useful for some specific operations like the provision of data to use for the POC experiment).
  \end{itemize}
\end{itemize}

\subsubsection{Swarm options}

Finally, few parameters about \textsc{Docker Swarm} can be set:

\begin{itemize}
  \item \texttt{image} (\emph{optional}): the image of \texttt{Docker Swarm} to use (default value: \texttt{swarm:1.2.5});
  \item \texttt{strategy} (\emph{optional}): the strategy of scheduling to give to \texttt{Docker Swarm} (default value: \texttt{spread}).
\end{itemize}


\newpage

% -*- root: Document.tex -*-

\subsection{Manage a cluster}
\label{subsec:clustersetup:manage}

We are assuming that your hosts are all reachable by SSH from the CLI.
You can ensure that by running:

\begin{lstlisting}[language=bash, basicstyle=\small]
$ ./remote.rb ping
\end{lstlisting}

Once the configuration of the hosts to use for creating the cluster has been given, here are the few steps needed to create a cluster:

\begin{itemize}
  \item \emph{Create nodes}: this command will configure \textsc{Docker} on the nodes to use in the \textsc{Docker Swarm} cluster, by resetting the configuration file of the \textsc{Docker} service of each host, like explained in Subsection~\ref{subsec:docker_config}:
    \begin{lstlisting}[language=bash, basicstyle=\small]
  $ ./remote.rb config-docker
    \end{lstlisting}
  \item \emph{Create the cluster}: this command will deploy the \textsc{Docker Swarm} cluster:
    \begin{lstlisting}[language=bash, basicstyle=\small]
  $ ./remote.rb create
    \end{lstlisting}
  \item \emph{Create an overlay network}: this command will create a \textsc{Dockr} overlay network:
    \begin{lstlisting}[language=bash, basicstyle=\small]
  $ ./remote.rb network
    \end{lstlisting}
\end{itemize}

All these commands can be run at once:

\begin{lstlisting}[language=bash, basicstyle=\small]
    $ ./remote.rb ping config-docker create network
\end{lstlisting}

You can ensure that your cluster have been well created by running:

\begin{lstlisting}[basicstyle=\small]
$ ./remote.rb test
\end{lstlisting}

If you want to recreate a cluster, use the command \texttt{remove} before, \textit{e.g.}:

\begin{lstlisting}[language=bash, basicstyle=\small]
    $ ./remote.rb remove
\end{lstlisting}


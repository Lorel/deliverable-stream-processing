% -*- root: Document.tex -*-

\section{Cluster Setup}
\label{sec:clustersetup}

A tiny CLI tool has been implemented to setup easily a \textsc{Docker Swarm} cluster and its overlay network.
This one can be found in the directory \texttt{remote} of the repository \securestreamrepo{}.
Once the configuration for the different hosts used to build the cluster is given (cf. Subsection~\ref{subsec:clustersetup:config}), the CLI can be used to build a \textsc{Docker Swarm} cluster (cf. Subsection~\ref{subsec:clustersetup:manage}).
Note that the example of configuration given here is present in the file \texttt{config.yml.example}: you can copy this file to \texttt{config.yml} and modify it with your own configuration.

When the CLI is called without any command, it returns the list of the available ones:

\begin{lstlisting}[basicstyle=\small]
$ ./remote.rb
Welcome to the remote manager !

Version: 0.1.0
Use one or several (you can chain them) of the following commands:

- help		 Print this usage notice
- version	 Print used version of Swarm
- ping		 Check if connexions for each VM are well configured
- create	 Create cluster
- network	 Create Docker overlay network
- hostnames	 Set node hostnames
- config-docker	 Upgrade Docker with the last release, and configure
- test		 Run hello world on each Docker node using the Swarm manager
- init-xp	 Initialize nodes for experimental POC by getting repository

For example:	 ./remote.rb ping config_docker create test network
\end{lstlisting}

\newpage

\input{8_2_1_Configuration}

\newpage

\input{8_2_2_Manage}

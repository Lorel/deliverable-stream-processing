% -*- root: Document.tex -*-

\chapter{Introduction}
\label{chap:introduction}

\citenote{
This work package aims at designing and implementing building blocks for developing big data
applications on top of microservices (WP3), themselves deployed within containers (WP2) in the
cloud. The main objectives are to make the development of cloud-based big data application easier,
safer, and faster. More specifically, the work package will result in the following outcomes:
T4.1: Secure and efficient (i.e., low latency and high throughput) communication mechanisms for
transmitting big data between microservices, and between clients and big data applications.
T4.2: A secure distributed key/value data store for big data application to store their data, and
used by the map/reduce framework of T4.4 to store (intermediary) computation results.
T4.3: Distributed scheduling mechanisms designed for executing computation tasks (running in
microservices) close to the data they depend on, and for placing data close to associated compute
tasks or to related data for better efficiency.
T4.4: A generic framework for map/reduce computations with big data across microservices, as
well as a collection of pre-defined components for big data processing.
This workpackage is led by UniNE, with significant technical contributions from TUD, IMP, and CC.
}


The data deluge imposed by a world of ever-connected devices, whose most emblematic example is the \emph{Internet of Things} (IoT), has fostered the emergence of novel data analytics and processing technologies to cope with the ever increasing \emph{volume}, \emph{velocity}, and \emph{variety} of information that characterize the big data era.
In particular, to support the continuous flow of information gathered by millions of IoT devices, data streams have emerged as a suitable paradigm to process flows of data at scale.
However, as some of these data streams may convey sensitive information, stream processing requires support for end-to-end security guarantees in order to prevent third parties accessing restricted data.

This document therefore introduces \SS{}, our initial work on a middleware framework for developing and deploying secure stream processing on untrusted distributed environments.
\SS{} supports the implementation, deployment, and execution of stream processing tasks in distributed settings, from large-scale clusters to multi-tenant Cloud infrastructures.
More specifically, \SS{} adopts a message-oriented~\cite{mom} middleware, which integrates with the SSL protocol~\cite{freier2011secure} for data communication and the current version of Intel{\textregistered}'s \emph{software guard extensions} (SGX)~\cite{costan_intel} to deliver end-to-end security guarantees along data stream processing stages.
\SS{} can scale vertically and horizontally by adding or removing processing nodes at any stage of the pipeline, for example to dynamically adjust according to the current workload.
The design of the \SS{} system is inspired by the dataflow programming paradigm~\cite{uustalu_essence_2005}: the developer combines together several independent processing components (\textit{e.g.}, mappers, reducers, sinks, shufflers, joiners) to compose specific processing pipes.
Regarding packaging and deployment, \SS{} smoothly integrates with industrial-grade lightweight virtualization technologies like Docker~\cite{docker}.

In this document, we propose the following contributions: (i)~we describe the design of \SS, (ii)~we provide details of our reference implementation, in particular on how to smoothly integrate our runtime inside an SGX enclave, and (iii) we perform an extensive evaluation with micro-benchmarks, as well as with a real-world dataset.

The remainder of the document is organized as follows.
Some related works to this topic are gathered in Chapter~\ref{chap:soa}.
To better understand the design of \SS, Chapter~\ref{chap:background} delivers a brief introduction to today's SGX operating mechanisms.
The architecture of \SS{} is then introduced in Chapter~\ref{chap:arch}.
The chapter~\ref{chap:proto} presents the implementation choices for a first prototype, and an example of a \SS{} program.
Chapter~\ref{chap:eval} discusses our extensive evaluation, presenting a detailed analysis of micro-benchmark performances, as well as more comprehensive macro-benchmarks with real-world datasets.
Finally, Chapter~\ref{chap:summ} briefly describes our future work, and Chapter~\ref{chap:summ} presents a user manual of the first \SS{} prototype.
